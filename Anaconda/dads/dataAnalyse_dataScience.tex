\begin{table}[h]
    \centering
    \begin{tabular}{|l|l|}
        \hline
        \textbf{Data Analyse} & \textbf{Science des Données} \\ \hline
        Apprentissage de Numpy & Apprentissage de Pandas \\ \hline
        Vecteurs & Les séries \\ \hline
        Matrices & Séries comme dictionnaire \\ \hline
        Tableau en 3D & Modifier une série Pandas \\ \hline
        La fonction "arange()" & Créer un DataFrame \\ \hline
        Les attributs de l'objet array & DataFrame depuis plusieurs séries \\ \hline
        La méthode reshape & DataFrame depuis une matrice NumPy \\ \hline
        La méthode resize & Modifier un DataFrame \\ \hline
        Concaténer les tableaux & Sélectionner les données d'un DataFrame \\ \hline
        Autoconstruction de tableaux & Sélectionner par tranche les données d'un DataFrame \\ \hline
        La fonction "linspace()" & Sélectionner conditionnellement les données d'un DataFrame \\ \hline
        arange() vs linspace() & Sélection des données d'un DataFrame selon plusieurs conditions \\ \hline
        Créer aléatoirement un tableau & Réinitialisation et modification des indices dans un DataFrame \\ \hline
        Méthodes de calcul & Créer un DataFrame multi-indexé \\ \hline
        Indexer des tableaux & Accéder aux données d'un DataFrame multi-indexé \\ \hline
        Indexer des tableaux binaires & Combiner les DataFrames \\ \hline
        Tableaux et opérateurs de comparaison & Grouper les données \\ \hline
        Modifier des tableaux selon des conditions & Tableaux croisées dynamiques \\ \hline
        Binariser une matrice & Opérations \\ \hline
        Opérations d'algèbre linéaire & Traiter les données manquantes \\ \hline
        Opération avec les matrices &  \\ \hline
    \end{tabular}
    \caption{Éléments de Data Analyse et Science des Données}
    \label{tab:data_analysis_science}
\end{table}